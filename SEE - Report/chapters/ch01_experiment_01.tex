%!TEX root = ../report.tex

\begin{document}
    \chapter{Experiment 01}
    \section{Deliverables}
    \begin{itemize}
        \item[] Write a report detailing your envisaged experimental setup, expected problems and expected performance. In your report, use terminology from the lecture (i.e. measurement system, measurand, measured quantity, and so forth) to describe your experiment. Your report should cover:
    \end{itemize}
    
    \begin{itemize}
        \item[1.] The relevant aspects of the design of the robot, especially how you mark the stop position and how you ensure identical start positions.
        \item[2.] An estimate of the expected precision of the to-be-observed data (i.e. the measurement process), including how you arrived at these estimates and why they are plausible.
        \item[3.] An estimate of the propagated orientation’s uncertainty (including the upper and lower bounds) caused by the errors in the measurement process, using the method of Jacobian error propagation.
    \end{itemize}
    
    \section{Design of Robot}
    {
    \subsection{Marking stop positions}
    \subsection{Ensuring identical start positions}
    }
    
    \section{Measurement Process}
    {
    \subsection{Estimate of Error}
    }
    
    \section{Error Propagation}
    {
    
    }
    
    
\end{document}
